%
% Florent JACQUET - Antonin Waltz
%
% Rapport de LO43 (A14)
%   SmallUTBMworld
%
%
\documentclass[a4paper]{report}
%packages
\usepackage[utf8]{inputenc}
\usepackage[francais]{babel}
\usepackage{graphicx}\graphicspath{{pictures/}}
\usepackage{float}
\usepackage[T1]{fontenc}
\usepackage{color}
\usepackage{fancyhdr}
\usepackage{listings}
\usepackage[colorlinks=true,allcolors=black]{hyperref}
\usepackage[font=small,labelfont=bf,margin=\parindent,tableposition=top]{caption}
\setcounter{tocdepth}{2}
\begin{document}
\begin{titlepage}
    \includegraphics[width=0.4\textwidth]{logo_utbm.png}
    \begin{center}
        \textsc{\LARGE Université de Technologie de Belfort Montbéliard}\\[1cm]
        \textsc{\Large LO43}\\
        \rule{\linewidth}{0.5mm}
        { \huge \bfseries Smallworld UTBM\\[0.4cm] }
        \rule{\linewidth}{0.5mm}
        \vskip1cm
        % Author and supervisor
        Florent \textsc{Jacquet}\\
        Antonin \textsc{Waltz}\\
        Superviseur: Amine \textsc{Ahmed Benyahia}\\
        \vskip1cm
        %\includegraphics[width=0.6\textwidth]{smallworld.png}
        \vfill
        {\large Automne 2014}
    \end{center}
\end{titlepage}
\newpage
\tableofcontents
\listoffigures
\newpage
\chapter{Présentation du projet}
\par
Durant ce semestre en LO43, nous avons choisi le projet \textit{Smallworld}
parmi tout ceux proposés.
\par
\textit{Smallworld} est un jeu de stratégie qui propose de gérer la destinée de
peuples fantastiques, chacun disposant d'une particularité unique. En plus de
ça, ceux-ci sont associés aléatoirement à un pouvoir spécial qui leur confére
une capacité supplémentaire.
\par
Le but est de conquérir des territoires afin d'obtenir le plus de points de
victoire. Cependant au fur et à mesure de l'avancé du jeu, un peuple a tendance
à s'affaiblir.  Le joueur qui le contrôle peut alors le faire passer en déclin
et choisir une nouvelle civilisation.
\par
Nous avons adapter les peuples et les pouvoirs à l'environnement de l'école, en
respectant les régles de base du jeu.
\newpage
\chapter{Organisation et répartition du travail}
\par
Notre groupe étant composé de 2 personnes, il était facile pour nous de
communiquer sur l'avancé du projet, la façon d'aborder telle où telle contrainte
technique où fonctionnelle ou tout simplement faire le point.
\par
Dans un premier temps et après nous être approprié \textit{Smallworld}, nous
avons réaliser les spécifications UML ensemble afin de partir sur la même base
de départ. Evidemment nous adapterons au fur et à mesure de l'implémentation en
fonction de l'évolution des contraintes. Nous nous sommes également mis d'accord
sur l'utilisation du pattern MVC, du polymorphisme, etc.
\par
Concernant les outils, nous avons utilisé \textbf{Eclipse}, \textbf{Dia} pour
l'UML et un dépôt \textbf{Git} pour la synchronisation des sources et le
versionnage. Nous avons également profiter des fonctionnalités de \textbf{Git}
pour effectuer un suivi efficace des fonctions à modifier où terminer.
\par
La répartition du travail s'est globalement fait ainsi, sachant que chacun
n'hésitait pas à intervenir dans les parties de l'autre si besoin était et qu'elle n'est pas définitive :
\begin{itemize}
    \item Florent : Modèle, Contrôleur, Rapport
    \item Antonin : Vue, Rapport
\end{itemize}
\newpage
\chapter{Architecture du projet}
\section{Première approche (Cas d'utilisation)}
Lorsque l'utilisateur lance le jeu, un premier menu s'offre à lui. Il peut choisir entre lancer une nouvelle partie, charger une partie déjà existante, afficher les règles où quitter le jeu.
\par Le seul cas qu'il est intéressant de détailler est celui d'une nouvelle partie. Après que le nombre de joueur ait été défini, les utilisateurs jouent chacun à tour de rôle. Ils ont alors le choix entre 4 actions :
\begin{itemize}
\item Acheter un peuple
\item Passer en déclin
\item Attaquer un terrain
\item Se redéployer
\end{itemize}
L'ordre et les différentes interactions entre ces actions sont détaillés dans le diagramme de séquence.
\section{Le modèle}
\par
C'est là le cœur du programme. Le modèle est une classe qui regroupe toutes les
classes du jeu en lui même, c'est à dire la carte, les jetons, les joueurs où
toute autre classe utile à la simulation du jeu par le programme.  En revanche, il ne s'agit
nullement de l'affichage, puisque cette tâche est laissée à la
partie Interface Utilisateur, la Vue.
\par
Cette partie est la seule pour laquelle nous avons réalisé un diagramme de
classe, étant donné qu'elle représente une grosse partie du projet, et que
beaucoup de classes entrent en jeu. Afin de mieux le structurer et de
savoir toujours dans quelle direction partir, le diagramme UML se trouvant en
Annexe 3 a été réalisé. Cela a permis d'avoiruimmédiatement un aperçu de
l'implémentation globale, et de se mettre d'accord sur la conception du projet.
\section{L'interface (la Vue)}
Lors d'un tour de jeu, le joueur doit pouvoir effectuer les actions qu'il désire. En même temps il est nécessaire qu'il ait accès à toutes les informations utiles.
\par
L'interface se divise en 4 parties distinctes :
\begin{itemize}
\item En haut, un récapitulatif des informations de base des joueurs adversaires : pseudo, peuple et pouvoir actif, icône d'avatar etc.
\item À droite, la liste des combinaisons peuple-pouvoir avec le nombre de pièces d'or correspondantes qu'il est possible de choisir.
\item Au centre, le plateau de jeu avec les territoires, le peuple par territoire, les attributs spéciaux etc.
\item En bas, les informations relatives au joueur actif, ainsi que les boutons d'action
\end{itemize}
Elle est cependant susceptible d'évoluer lors de l'implémentation.
\section{Le déroulement d'une phase de jeu (Séquence)}
\par
Cette partie est très importante, puisqu'il s'agit en somme des règles du jeu.
Étant donné que le programme est évènementiel, il suffit de savoir
quelle action peut être faite à quel moment, et cela se résume dans le
diagramme de séquence que nous avons fait, qui se trouve en Annexe 2.
\par
Ce diagramme s'étend sur la séquence de jeu d'un joueur, ce qui ne
représente pas la séquence complète du programme, mais simplement la partie
essentielle de celle-ci. On omet donc les étapes de fin de tour, ainsi que celles
de l'initialisation et de la fin du jeu, qui ne sont pas \"essentielles\" au déroulement
d'une partie.
\newpage
\chapter{Implémentation}
Étant donné que l'implémentation n'a pas été complétement terminée, nous la détaillerons dans la version finale du rapport.
\newpage
\chapter{Annexes}
\section{Diagramme des cas d'utilisation}
\begin{figure}[H]
    \begin{center}
        \includegraphics[width=0.85\textwidth]{use_case.png}
        \caption{Diagramme des cas d'utilisation du \textit{Smallworld}}
    \end{center}
\end{figure}
\section{Diagramme de séquence}
\begin{figure}[H]
    \begin{center}
        \includegraphics[width=0.85\textwidth]{sequence.png}
        \caption{Diagramme de séquence d'un tour de jeu}
    \end{center}
\end{figure}
\section{Diagramme de classes}
\begin{figure}[H]
    \begin{center}
        \includegraphics[width=0.85\textheight,angle=90]{classe.png}
        \caption{Diagramme de classe du modèle}
    \end{center}
\end{figure}
\section{Interface}
\begin{figure}[H]
    \begin{center}
        \includegraphics[width=0.85\textheight,angle=90]{plateau.jpg}
        \caption{Maquette de l'interface lors d'un tour de jeu}
   \end{center}
\end{figure}
\end{document}

